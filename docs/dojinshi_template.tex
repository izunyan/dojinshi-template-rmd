% Options for packages loaded elsewhere
\PassOptionsToPackage{unicode}{hyperref}
\PassOptionsToPackage{hyphens}{url}
%
\documentclass[
  xelatex,ja=standard, b5paper]{bxjsbook}
\author{}
\date{\vspace{-2.5em}}

\usepackage{amsmath,amssymb}
\usepackage{lmodern}
\usepackage{iftex}
\ifPDFTeX
  \usepackage[T1]{fontenc}
  \usepackage[utf8]{inputenc}
  \usepackage{textcomp} % provide euro and other symbols
\else % if luatex or xetex
  \usepackage{unicode-math}
  \defaultfontfeatures{Scale=MatchLowercase}
  \defaultfontfeatures[\rmfamily]{Ligatures=TeX,Scale=1}
\fi
% Use upquote if available, for straight quotes in verbatim environments
\IfFileExists{upquote.sty}{\usepackage{upquote}}{}
\IfFileExists{microtype.sty}{% use microtype if available
  \usepackage[]{microtype}
  \UseMicrotypeSet[protrusion]{basicmath} % disable protrusion for tt fonts
}{}
\makeatletter
\@ifundefined{KOMAClassName}{% if non-KOMA class
  \IfFileExists{parskip.sty}{%
    \usepackage{parskip}
  }{% else
    \setlength{\parindent}{0pt}
    \setlength{\parskip}{6pt plus 2pt minus 1pt}}
}{% if KOMA class
  \KOMAoptions{parskip=half}}
\makeatother
\usepackage{xcolor}
\IfFileExists{xurl.sty}{\usepackage{xurl}}{} % add URL line breaks if available
\IfFileExists{bookmark.sty}{\usepackage{bookmark}}{\usepackage{hyperref}}
\hypersetup{
  hidelinks,
  pdfcreator={LaTeX via pandoc}}
\urlstyle{same} % disable monospaced font for URLs
\usepackage{longtable,booktabs,array}
\usepackage{calc} % for calculating minipage widths
% Correct order of tables after \paragraph or \subparagraph
\usepackage{etoolbox}
\makeatletter
\patchcmd\longtable{\par}{\if@noskipsec\mbox{}\fi\par}{}{}
\makeatother
% Allow footnotes in longtable head/foot
\IfFileExists{footnotehyper.sty}{\usepackage{footnotehyper}}{\usepackage{footnote}}
\makesavenoteenv{longtable}
\usepackage{graphicx}
\makeatletter
\def\maxwidth{\ifdim\Gin@nat@width>\linewidth\linewidth\else\Gin@nat@width\fi}
\def\maxheight{\ifdim\Gin@nat@height>\textheight\textheight\else\Gin@nat@height\fi}
\makeatother
% Scale images if necessary, so that they will not overflow the page
% margins by default, and it is still possible to overwrite the defaults
% using explicit options in \includegraphics[width, height, ...]{}
\setkeys{Gin}{width=\maxwidth,height=\maxheight,keepaspectratio}
% Set default figure placement to htbp
\makeatletter
\def\fps@figure{htbp}
\makeatother
\setlength{\emergencystretch}{3em} % prevent overfull lines
\providecommand{\tightlist}{%
  \setlength{\itemsep}{0pt}\setlength{\parskip}{0pt}}
\setcounter{secnumdepth}{5}
\makeatletter
\def\emptypage@emptypage{%
    \hbox{}%
    \thispagestyle{headings}%
    \newpage%    
}%
\def\cleardoublepage{%
        \clearpage%
        \if@twoside%
            \ifodd\c@page%
                % do nothing
            \else%
                \emptypage@emptypage%
            \fi%
        \fi%
    }%
\makeatother
\ifLuaTeX
  \usepackage{selnolig}  % disable illegal ligatures
\fi

\begin{document}

{
\setcounter{tocdepth}{1}
\tableofcontents
}
\hypertarget{hajimeni}{%
\chapter*{はじめに}\label{hajimeni}}
\addcontentsline{toc}{chapter}{はじめに}

本テンプレートは同人誌の原稿を書くためのテンプレートです。

\hypertarget{ux5b9fux884cux65b9ux6cd5}{%
\section{実行方法}\label{ux5b9fux884cux65b9ux6cd5}}

\begin{itemize}
\tightlist
\item
  BuildタブでBuild Book \textgreater{} bookdown::pdf\_book
\item
  Knitでやるとエラーになるよ
\end{itemize}

\hypertarget{ux3068ux3066ux3082ux304aux4e16ux8a71ux306bux306aux3063ux305fux30b5ux30a4ux30c8}{%
\section{とてもお世話になったサイト}\label{ux3068ux3066ux3082ux304aux4e16ux8a71ux306bux306aux3063ux305fux30b5ux30a4ux30c8}}

\begin{itemize}
\tightlist
\item
  \href{https://teastat.blogspot.com/2019/01/bookdown.html}{Bookdownによる技術系同人誌執筆}
\end{itemize}

\hypertarget{haiiro}{%
\part{灰色文字の区切り}\label{haiiro}}

\hypertarget{setting}{%
\chapter{設定}\label{setting}}

\hypertarget{setting_japanese}{%
\section{日本語}\label{setting_japanese}}

\hypertarget{caution}{%
\chapter{注意事項}\label{caution}}

\begin{itemize}
\tightlist
\item
  奇数ページだと1ページ白紙になるので,偶数ページにしないとだめ
\end{itemize}

\clearpage
\vspace*{\stretch{1}}
\begin{flushright}
\begin{minipage}{0.5\hsize}
\begin{description}
  \item{著者:} 著者名
  \item{発行:} 2019年11月18日
  \item{サークル名:} サークル名
  \item{連絡先:} メールアドレス
  \item{印刷:} 印刷所名
\end{description}
\end{minipage}
\end{flushright}
\clearpage

\end{document}
